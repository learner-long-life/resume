% Paul Pham's Curriculum Vitae

\documentclass[letter]{article}
\usepackage{fancyhdr}
\usepackage[osf]{mathpazo}
\usepackage[pdftex]{graphicx,color,hyperref}

\pagestyle{fancyplain}
\lhead[\fancyplain{}{\bfseries\thepage}]
    {\fancyplain{}{\bfseries\rightmark}}
\rhead[\fancyplain{}{\bfseries\leftmark}]
    {\fancyplain{}{Paul T. Pham \hfill\bfseries\thepage}}
\cfoot[]{}
\addtolength{\headheight}{1.6pt}

\addtolength{\topmargin}{-0.75in}	% repairing LaTeX's huge margins...
\addtolength{\textwidth}{1.1in}	% same here....
\setlength{\headwidth}{\textwidth}
\setlength{\parindent}{0mm}	% indented paragraphs just don't make it
			   	% on a resume.  (My opinion; if you
			   	% disagree, give this whatever value you
			   	% want.) 
\setlength{\textheight}{9.5in}	% to fit more resume on the page

\begin{document}

\thispagestyle{empty}           % no headers on the first page

\reversemarginpar		% this puts the margin notes on the
				% left-hand side of the resume.

{\LARGE {\bf Paul T. Pham}}
\par
\vspace{.25in}
E-mail: \href{mailto:paul.tan.the.pham@gmail.com}{paul.tan.the.pham@gmail.com}
\hspace*{\fill}
3311 6th Ave SW, Apt A311
\linebreak
Phone: (253) 235-9025
\hspace*{\fill}
Olympia, WA 98502

				% address stuff.  The \hspace*{\fill} is
				% necessary to get things laid out
				% correctly.  The sizes fed to \vspace are
				% variable, depending on how much
				% whitespace you want, and how much
				% stuff you're trying to fit on a page....
\par
\vspace{.25in}
				% if the \marginpar is at the beginning
				% of the line, it loses.  *sigh*.  This
				% is a typical entry in a resume.   (I'm
				% defining an entry as something
				% important enough to have a marginal note
				% setting it off.)  Give it some space
				% with \vspace, use \par to break
				% between paragraphs, enter your text,
				% and place the \marginpar at the end.

%%%%%%%%%%%%%%%%%%%%%%%%%%%%%%%%%%%%%%%%%%%%%%%%%%%%%%%%%%%%%%%%%%%%%%%%%%%%
\vspace{\baselineskip}
{\bf {Etsy}} \hfill Brooklyn, New York\marginpar{{\bf Software}}
\par
{\em Senior Software Engineer, Search Infrastructure} \hfill May 2016---January 2018\marginpar{{\bf Engineering}}
\par
Manager: Gordon Radlein
\vspace{0.5\baselineskip}
\par
Technologies: Java, Shell Scripting, Python, Docker, Lucene, ElasticSearch, Linux, Grafana, PagerDuty, Continuous Integration
\vspace{0.5\baselineskip}
\par
Design and implement distributed, highly-available infrastructure for product search that powers Etsy. Earned \$700k+ in additional annualized revenue by running a UI experiment with A/B tests. Tech lead for a new search indexing / inference pipeline.

\vspace{\baselineskip}
\par
{\bf {D-Wave Systems, Inc.}} \hfill Burnaby, British Columbia
\par
{\em Software Development Engineer, Retail Customer Experience} \hfill January 2008---June 2009
\par
Manager: Paula Gil
\vspace{0.5\baselineskip}
\par
Technologies: Python, MATLAB, Shell Scripting
\vspace{0.5\baselineskip}
\par
Accelerated machine learning algorithms using a GPU and Theano. Automated testing and training of models.

\vspace{\baselineskip}
\par
{\bf {Amazon.com}} \hfill Seattle, Washington
\par
{\em Software Development Engineer, Retail Customer Experience} \hfill January 2008---June 2009
\par
Manager: Doug Irvine
\vspace{0.5\baselineskip}
\par
Technologies: Java, J2EE, Tomcat, Spring, jQuery, Selenium
\vspace{0.5\baselineskip}
\par
Maintained a large-scale, high-availability
retail website. Implemented a pipeline for customers to write product reviews.

%%%%%%%%%%%%%%%%%%%%%%%%%%%%%%%%%%%%%%%%%%%%%%%%%%%%%%%%%%%%%%%%%%%%%%%%%%%%%%

{\bf The Evergreen State College} \hfill Olympia, Washington
\marginpar{{\bf Teaching}}

\par
{\em Member of the Faculty}\hfill 2013-2014,2023-Present
\marginpar{{\bf Experience}}

\vspace{0.5\baselineskip}
\par
Data Structures \& Algorithms\hfill Fall 2023, Fall 2024
\vspace{0.5\baselineskip}
\par
Web Engineering, Software Construction, AI Self-Hosting\hfill Winter, Spring 2024
\vspace{0.5\baselineskip}
\par
Java Programming I, II, III\hfill 2013-2014


{\bf CodeFellows} \hfill Seattle, Washington

\par
{\em Instructor} \hfill March 2015---October 2015

\vspace{0.5\baselineskip}
\par
\href{https://www.codefellows.org/courses/foundations-1/computer-science-and-web-development}{Foundations of Computer Science and Web Development}\hfill April---May 2015
\vspace{0.5\baselineskip}
\par
\href{https://www.codefellows.org/courses/foundations-2/python}{Python Foundations II}\hfill March---October 2015

%%%%%%%%%%%%%%%%%%%%%%%%%%%%%%%%%%%%%%%%%%%%%%%%%%%%%%%%%%%%%%%%%%%%%%%%%%%%
\vspace{0.5\baselineskip}
{\bf University of Washington} \hspace*{\fill}September 2005---December 2006
\marginpar{{\bf Education}}
\par
Ph.D. in Computer Science \hspace*{\fill}March 2010---Present\\
Thesis: \href{https://dl.acm.org/doi/book/10.5555/2604498}{Low-depth quantum architectures for factoring.} Advised by Aram Harrow, Dave Bacon\\
\vspace{0.5\baselineskip}

{\bf Massachusetts Institute of Technology}
\par
M.Eng. in Electrical Engineering \& Computer Science \hspace*{\fill}February 2005.\\
Thesis: \href{https://citeseerx.ist.psu.edu/viewdoc/summary?doi=10.1.1.120.6651&rank=1}{A general-purpose pulse sequencer for quantum computing.}{} Advised by Isaac Chuang\\
\par
Bachelor of Science in Electrical Engineering and Computer Science, June 2004.
\vspace{\baselineskip}
\par
				% This is a slightly more complicated
				% example.  \vspace gets a `doublespace'
				% (1.5\baselineskip) between entries, and
				% a  `singlespace' between paragraphs of
				% an entry.
				% (1.5\baselineskip means twice the value
				% of \baselineskip).  
%%%%%%%%%%%%%%%%%%%%%%%%%%%%%%%%%%%%%%%%%%%%%%%%%%%%%%%%%%%%%%%%%%%%%%%%%%%%%%
\vspace{\baselineskip}
\par
{\bf {Private Voting}} \marginpar{{\bf Open}}
\par
{\em Project Admin, Lead Developer} \hfill January 2005---Present\marginpar{{\bf Source}}
\par
\url{http://github.com/invisible-college/democracy}
\par
Developed a private trading system for zero-knowledge assets
using AZTEC and Ethereum. Designed architecture, scoped out
requirements, and managed other designers and developers
towards a product launch.

\vspace{\baselineskip}
\par
{\bf {Pulse Programmer for Quantum Computing}}
\par
{\em Project Admin, Lead Developer} \hfill January 2005---Present
\par
\url{http://pulse.sf.net}
\par
Built an open source reconfigurable radio-frequency signal generator
for quantum information processing experiments.
In use at eight experimental trapped ion research groups around the world.

%%%%%%%%%%%%%%%%%%%%%%%%%%%%%%%%%%%%%%%%%%%%%%%%%%%%%%%%%%%%%%%%%%%%%%%%%%%%%%
\vspace{\baselineskip}
\par
``Introductory programming meets the real world: using real problems and data in CS1.''
\marginpar{{\bf Publications}}\\
Ruth Anderson, Michael D. Ernst, Robort Ordo\~{n}ez, \textbf{Paul Pham}, Steven A. Wolfram.\\
\url{https://dl.acm.org/citation.cfm?id=2538994}\\
SIGCSE 2014, Proceedings of the 45th ACM technical symposium on computer science education.

\vspace{\baselineskip}
\par
``A 2D nearest-neighbor quantum architecture for factoring.''\\
\textbf{P. Pham}, K.M. Svore. \hfill \url{http://arxiv.org/abs/1207.6655}\\
Reversible Computation Workshop, June 2012 \hfill Copenhagen, Denmark

\vspace{\baselineskip}
\par
``Component-based invisible computing.''\\
A. Forin, J. Helander, \textbf{P. Pham}, J. Rajendiran.\\
IEEE Realtime Embedded Systems Workshop, December 2001 \hfill London, UK

%%%%%%%%%%%%%%%%%%%%%%%%%%%%%%%%%%%%%%%%%%%%%%%%%%%%%%%%%%%%%%%%%%%%%%%%%%%%%%

\vspace{\baselineskip}

``\href{https://patents.justia.com/patent/9779359}{Quantum arithmetic on two-dimensional quantum architectures.}''
\marginpar{{\bf Patents}}\\
\textbf{P. Pham}, K. Svore.\\
U.S. Patent No. 9779359.\\
Filed: March 14, 2012\\
Date of Patent: October 3, 2017\\

\vspace{0.5\baselineskip}

``\href{https://patents.google.com/patent/US7246353}{Method and system for managing the execution of threads and data processing.}''\\
A. Forin, J. Helander, \textbf{P. Pham}.\\
U.S. Patent No. 7246353B2.\\
Filed on June 12, 2002.
Date of Patent: July 17, 2007\\


%%%%%%%%%%%%%%%%%%%%%%%%%%%%%%%%%%%%%%%%%%%%%%%%%%%%%%%%%%%%%%%%%%%%%%%%%%%%%%
\vspace{\baselineskip}

{\bf MIT ACM/IEEE Programming Competition} \hfill Cambridge, Massachusetts
\marginpar{{\bf Community}}

{\em Contest Chair, Lead Developer, Organizer} \hfill 2001-2003\marginpar{{\bf Building}}
\vspace{0.5\baselineskip}
\par

\url{http://www.battlecode.org}\\
\url{http://web.mit.edu/ieee/6.370/2003/web/}\\
Created a long-running
programming competition for real-time strategy artificial intelligence
(AI) agents, which matches winning student contestants with corporate
sponsors that now include
Dropbox, Blizzard Entertainment, Amazon, Google, Oracle, D.E. Shaw, Akamai.

\pagebreak

%%%%%%%%%%%%%%%%%%%%%%%%%%%%%%%%%%%%%%%%%%%%%%%%%%%%%%%%%%%%%%%%%%%%%%%%%%%%%%
\vspace{\baselineskip}
\par
{\bf {Microsoft Research}} \hfill Seattle, WA\marginpar{{\bf Research}}
\par
{\em Research Intern} \hfill June---August 2011\marginpar{{\bf Experience}}
\par
\href{http://research.microsoft.com/en-us/groups/quarc/default.aspx}{Quantum Architectures and Computation Group}
\par
Mentor: Krysta Svore
\par
Designed a 2D nearest-neighbor quantum architecture for period-finding with
depth $O(L \log L)$ for factoring an $L$-bit integer. Pending patent
application.

\vspace{\baselineskip}
\par
{\bf {University of Washington Dept. of Physics and Astronomy}} \hfill Seattle, WA
\par
{\em Graduate Research Assistant} \hfill January---July 2007, May---June 2010
\par
\href{http://depts.washington.edu/qcomp/}{Trapped Ion Quantum Computing Group}
\par
Advisor: Prof. Boris Blinov
\par
Built a programmable radio-frequency system for ion trap control including
photomultiplier tube input counting.

\par

\vspace{\baselineskip}
\par
{\bf Max Planck Institute for Quantum Optics} \hfill Garching, Germany
\par
{\em Visiting Ph.D. Student} \hfill July 2005---August 2005
\par
\href{http://www.mpq.mpg.de/qsim/}{Quantum Analog Simulation Group}
\par
Advisor: Dr. Tobias Sch\"atz
\par
Built a programmable radio-frequency system for ion trap control with phase-coherent
frequency-switching.

\par

\vspace{\baselineskip}
\par
{\bf University of Innsbruck} \hfill Innsbruck, Austria
\par
{\em Visiting Ph.D. Student} \hfill February 2005---June 2005
\par
\href{http://heart-c704.uibk.ac.at/index.html}{Quantum Optics and Spectroscopy Group}
\par
Advisor: Univ. Prof. Rainer Blatt
\par
Built a programmable radio-frequency system for ion trap control with shaped amplitudes.
\par

\vspace{\baselineskip}
\par
{\bf MIT Center for Bits and Atoms} \hfill Cambridge, Massachusetts
\par
{\em Graduate Research Assistant} \hfill September 2003---January 2005
\par
\href{http://web.mit.edu/~cua/www/quanta/}{quanta Research Group}
\par
Advisor: Prof. Isaac Chuang
\par
Designed and built instrumentation for quantum computing experiments.
\par

\vspace{\baselineskip}
\par
{\bf Microsoft Research} \hfill Redmond, WA
\par
{\em Research Intern} \hfill June 2001---September 2001
\par
\href{http://research.microsoft.com/en-us/projects/mic/default.aspx}{Invisible Computing Group}\hfill June 2003---August 2003
\par
Mentors: Alessandro Forin, Johannes Helander
\par
Added work items to the scheduler of an embedded real-time kernel.
Designed and assembled the electronics for a wireless sensor demo.

\pagebreak

%%%%%%%%%%%%%%%%%%%%%%%%%%%%%%%%%%%%%%%%%%%%%%%%%%%%%%%%%%%%%%%%%%%%%%%%%%%%%%
{\bf Gordon Radlein}
\marginpar{{\bf References}}\\
Senior Director of Engineering\\
Etsy\\
117 Adams St\\
Brooklyn, NY 11237\\
E-mail: gordon@etsy.com

\par
\vspace{\baselineskip}
{\bf Krysta Svore}\\
Researcher\\
Microsoft Research\\
1 Microsoft Way\\
Redmond, WA 98052\\
Phone: (425) 421-6996\\
E-mail: \url{ksvore@microsoft.com}\\

\end{document}




